% Configuracion del documento
%

\usepackage[utf8]{inputenc} % Required for inputting international characters
\usepackage[T1]{fontenc} % Output font encoding for international characters

\usepackage[spanish,es-tabla]{babel} % OJO

\usepackage{graphicx} % Logo

%
% La fuente a utilizar
%
%\usepackage{mathpazo} % Palatino font
% Tomado de la plantilla propuesta
\usepackage[defaultfam,tabular,lining]{montserrat} %% Option 'defaultfam'

%
% Algunos paquetes
%
\usepackage{amsmath}
\usepackage{amssymb}
\usepackage{amsthm}
\usepackage{enumitem}
\usepackage{bm}
\usepackage{csquotes}
\usepackage{imakeidx}
\usepackage{hyperref}
\usepackage{wallpaper}
\usepackage[left=2cm,top=2.5cm,right=2cm,bottom=2.5cm]{geometry}
\usepackage{fancyhdr}
\decimalpoint

% Teoremas, definiciones, etc.
\newtheorem{definition}{Definición}
\newtheorem{theorem}{Teorema}
\newtheorem{lemma}{Lema}
\newtheorem{proposition}{Proposición}

% Color. Tomado de la plantilla propuesta
% De la plantilla docx. Blue, Accent 1, Darker 25%: 47 84 150, #2f5496
%\definecolor{blueunadm}{RGB}{3, 113, 192}
\definecolor{blueunadm}{RGB}{47, 84, 150}

% El título del índice
% https://tex.stackexchange.com/a/28518
%\renewcommand{\contentsname}{\centering\fontsize{12}{18}\selectfont{\textcolor{blueunadm}{\textbf{ÍNDICE}}}}
\addto\captionsspanish{
    \renewcommand{\contentsname}{\centering\fontsize{12}{18}\selectfont{\textcolor{blueunadm}{\textbf{ÍNDICE}}}}
}

% URL
\hypersetup{
  colorlinks=true,
  linkcolor=blueunadm,
  filecolor=magenta,
  urlcolor=blue,
  citecolor=blue,
  pdfauthor=\estudiante,
  pdftitle={\estudiante. \uaprendizaje. \actividad},
  pdfsubject={\udidactica, \actividad},
  pdfkeywords={\udidactica, \actividad},
}
\urlstyle{same}

%
% extras
%
\author{\estudiante}
\title{\udidactica \uaprendizaje \actividad}

%
% El encabezado de cada una de las páginas
%
\fancyhf{}

% Elimina la línea en el encabezado
% https://tex.stackexchange.com/a/105888
\renewcommand{\headrulewidth}{0pt}

% Ajustes para la imagen del encabezado
\setlength{\headheight}{84pt}
\addtolength{\topmargin}{-45pt}

% La imagen del encabezado
%https://tex.stackexchange.com/a/322163
\chead{\makebox[\headwidth][c]{\includegraphics[width=\paperwidth]{imagenes/encabezado}}}
\pagestyle{fancy}


% La bibliografia, por medio de Biber
\usepackage[
        backend=biber,
        citestyle=authoryear,
        style=apa
]{biblatex}
\addbibresource{\referencias}

\makeindex
